%Paquetes para idioma español y codifcación UTF8
\usepackage[spanish]{babel}
\usepackage[utf8]{inputenc}
\usepackage{csquotes}

%%% BIBLATEX
\usepackage{biblatex}
%%% BIBLIOGRAPHY
\addbibresource{references.bib}

%paquete de matemática, vuela
\usepackage{amsmath}
%fuente 'fourier'
\usepackage{fourier}
%paquete para URLs
\usepackage{url}
%paquete para ubicar las imágenes
\usepackage{float}
%paquete para imágenes y en dónde las tiene que buscar
\usepackage{graphicx}
\graphicspath{{images/}}
%paquete para epígrafes
\usepackage{subcaption}
%paquete para definir los márgenes de la hoja
\usepackage[left=1.5cm,right=1.5cm,top=3cm,bottom=3cm]{geometry}
%paquete para poner todos y comentarios
\usepackage[colorinlistoftodos]{todonotes}
%paquete para trabajar con código
\usepackage{listings}
%paquete para trabajar con colores y definir propios
\usepackage{color}

%Cabeceras
\usepackage{fancyhdr}
